\chapter{Conclusiones y trabajo a futuro} \label{section-conclusiones}
\markright{Conclusiones}

\paragraph{}En este capítulo se presentan las conclusiones del trabajo realizado en este proyecto de grado y las principales líneas de trabajo a futuro.

\section{Conclusiones}

\paragraph{}En este proyecto se presentó un estudio comparativo entre dos tipos de clasificadores de aprendizaje automático, SVM y redes neuronales, para el problema de \textit{HCSP} en el marco del paradigma de Savant Virtual. Se entrenaron diferentes configuraciones de redes neuronales y se compararon las soluciones obtenidas por cada uno de los clasificadores en términos de su \textit{makespan}, precisión y decisiones de selección de máquinas frente a errores. Las redes neuronales fueron entrenadas utilizando distintas funciones de activación, así como también variando la cantidad de capas ocultas, manteniendo los demás parámetros de configuración constantes. De esta manera se generaron 9 redes neuronales de 2, 3 y 4 capas ocultas para las funciones de activación \textit{tanh}, \textit{relu} e \textit{identity} respectivamente. Fueron utilizadas para el entrenamiento 100 instancias del problema de 512 tareas y 16 máquinas, lo que se traduce en 512000 instancias de entrenamiento. Se profundizó el estudio para aquellas redes neuronales que mostraron mejores resultados en cuanto al \textit{makespan} obtenido.

\paragraph{}El análisis experimental fue realizado clasificando instancias del problema de diferentes dimensiones, desde 17 tareas y 16 máquinas hasta 1024 tareas y 16 máquinas, con el fin de analizar el comportamiento de los clasificadores para instancias más pequeñas, de igual y mayor tamaño en comparación a los datos utilizados durante el entrenamiento, de 512 tareas y 16 máquinas. Para cada dimensión del problema se utilizaron 10 instancias del problema como instancias de validación y se calculó el \textit{makespan} obtenido mediante la clasificación con las redes neuronales, así como la precisión y el porcentaje de selección de máquinas más rápidas frente a un error para el promedio de las 10 instancias del problema.

\paragraph{}Los resultados experimentales muestran que las redes neuronales de 2 capas ocultas generaron soluciones con un menor \textit{makespan} que aquellas con 3 y 4 capas ocultas; esto es una característica común a todas las redes neuronales utilizadas sin importar su función de activación. Además, el makespan tiende a ser menos variable para estas redes neuronales. Todo esto es importante dado que el \textit{makespan} se entiende como la métrica fundamental del éxito de una solución generada para el problema.
% TODO explain -> utilizando menor tiempo de entrenamiento

\paragraph{}En comparación con SVM, las redes neuronales de dos capas ocultas con funciones de activación \textit{tanh} e \textit{identity} muestran mejoras en \textit{makespan} para dimensiones grandes. Para dimensiones de a partir de 400 tareas y 16 máquinas se comienzan a observar mejoras en el \textit{makespan} para los resultados obtenidos con ambas redes neuronales. En particular, la red neuronal de 2 capas ocultas con activación \textit{tanh}, mejora el \textit{makespan} para dimensiones grandes, teniendo una mejor precisión en clasificación que SVM y teniendo un porcentaje de selección más bajo de máquinas más rápidas frente a errores que SVM. En cuanto a la red neuronal de dos capas ocultas con función de activación \textit{identity} se observa que la precisión en clasificación es muy similar a la precisión en clasificación de SVM, con menos variabilidad, al igual que el porcentaje de selección de máquinas más rápidas, también mostrando leves mejoras en el \textit{makespan} de las soluciones. Para las redes neuronales de dos capas ocultas entrenadas con la función de activación \textit{relu} los resultados en cuanto a \textit{makespan} no mejoran el \textit{makespan} obtenido por SVM, aunque la precisión en clasificación es levemente mayor que la precisión en clasificación de SVM, teniendo un porcentaje de selección de mejores máquinas frente a un error menor que el de SVM. 

\section{Trabajo a futuro}

\paragraph{}A continuación se detallan las principales lineas de trabajo a futuro que surgen del trabajo realizado para el proyecto de grado.

\paragraph{}Las instancias del problema \textit{HCSP} utilizadas para el entrenamiento y prueba de los clasificadores tienen como característica una baja heterogeneidad de tareas y de máquinas, lo cual hace difícil la aplicación de dichos clasificadores a la resolución de instancias reales del problema, donde se presume que lo corriente es tener tareas heterogéneas ejecutando en máquinas de características homogéneas o heterogéneas. Dentro de las líneas de trabajo a futuro se incluye el entrenamiento y prueba de clasificadores utilizando instancias del problema de mayor heterogeneidad en tareas y máquinas, siendo de interés analizar el comportamiento de aquellas configuraciones de redes neuronales que durante este trabajo mostraron resultados más prometedores. También es de interés probar distintas configuraciones parámetros de las redes neuronales, ya que durante este trabajo, se analizaron redes neuronales entrenadas con diferentes arquitecturas y funciones de activación, dejando el resto de los parámetros de configuración con sus valores predeterminados. 

\paragraph{}Por otro lado, se considera necesario extender este estudio comparativo aplicando búsquedas locales tanto a las soluciones generadas con redes neuronales como a aquellas generadas por SVM, para evaluar el efecto que esto pudiera tener en la calidad de las mismas, ya sea en el marco de una implementación de MapReduce o no.

\paragraph{}Finalmente, también es de interés realizar pruebas con otros clasificadores de aprendizaje automático realizando estudios comparativos con los resultados ya obtenidos para el modelo de SV y eventualmente probar el enfoque de SV para la resolución de problemas reales.
% TODO discutir, tema sensible -> Un ejemplo de problema real sobre el cual se puede probar el enfoque es el problema de asignación de contenedores de basura en una ciudad para la optimización de las rutas de los camiones que se encargan de vaciar y mantener los contenedores.
