\chapter{Descripción del problema} \label{section-descripcion-problema}
\markright{Descripción del problema}

\paragraph{}Este capítulo presenta una introducción al problema HCSP basada fuertemente en el trabajo de \citet{bib-doctorado-nesmachnow}. Este problema es abordado en el marco de este proyecto y el paradigma de Savant Virtual. En la sección \ref{section:descripcion-problema,subsection:introduccion} se introduce al problema en su contexto, en la sección \ref{section:descripcion-problema,subsection:formulacion-problema} se presenta la formulación del problema, en la sección \ref{section:descripcion-problema,subsection:complejidad-computacional} se analiza la complejidad computacional del HCSP, en la sección \ref{section:descripcion-problema,subsection:estimacion-tiempo-ejecucion} se discute la estimación de tiempos de ejecución utilizada en las instancias del problema, en la sección \ref{section:descripcion-problema,subsection:instancias-del-problema} se analiza la estructura de las instancias del problema utilizadas durante este Proyecto de Grado, y finalmente en la sección \ref{section:descripcion-problema,subsection:heuristica} se
presenta la heurística de referencia utilizada a la hora de llevar a cabo el aprendizaje automático.

\section{Introducción} \label{section:descripcion-problema,subsection:introduccion}

% Historia de HCSP
 
\paragraph{}En un contexto donde el poder de cómputo se ve incrementado constantemente y la comercialización de computadores de bajo costo es común, se vuelve común el uso de ambientes computacionales distribuidos basados en componentes heterogéneos para la resolución de problemas complejos y con diversas aplicaciones. Esto incluye a aquellos problemas que por escala o limitaciones de recursos no eran resolubles anteriormente y a aquellos problemas surgidos por la utilización de dichos sistemas heterogéneos y distribuidos. 

\paragraph{}Uno de los problemas fundamentales que surgen al emplear computación heterogénea (o HC, del inglés \textit{Heterogeneous Computing}) es el de encontrar una manera de planificar un conjunto de tareas a ser ejecutadas tomando en cuenta las características y poder computacional de cada uno de los elementos que conforman al sistema. El objetivo de este problema de planificación es el de asignar tareas de manera óptima a recursos computacionales optimizando alguna métrica de eficiencia. Usualmente se tiende a utilizar estrategias que optimicen el \textit{makespan}, una métrica que contempla el tiempo total de ejecución invertido hasta completar todas las tareas del sistema.

\paragraph{}Los problemas de planificación en sistemas multiprocesador han sido ampliamente estudiados en el campo de la investigación operativa, y numerosos métodos han sido propuestos para generar planificaciones precisas en tiempos razonables (\citet{bib-rewini-scheduling}, \citet{bib-leung-handbook}). En su formulación clásica, los problemas de planificación asumen un entorno computacional compuesto por recursos homogéneos. Sin embargo, en la decada de 1990 la comunidad de investigadores empezó a prestarle atención a problemas en entornos HC, dada la popularización de la computación distribuida y el uso cada vez más frecuente de clusters heterogéneos (\citet{bib-freund}, \citet{bib-eshaghian-heterogeneous}).

\paragraph{}Los problemas tradicionales de planificación son NP-difíciles (\citet{bib-garey}), por lo cual los métodos clásicos y exactos sólo son útiles para resolver instancias reducidas del problema, dado que su poca eficiencia hace que sea inviable aplicarlos para problemas de gran dimensión con tiempos razonables de ejecución. Al tratar con ambientes computacionales de grandes dimensiones, las heurísticas ad-hoc y metaheurísticas se han destacado como métodos prometedores para resolver problemas de HC. Aunque estos métodos no garantizan obtener una solución óptima para el problema, obtienen planificaciones cercanas a las óptimas que usualmente satisfacen los requerimientos de eficiencia para escenarios reales, con tiempos razonables.
Entre un conjunto amplio de técnicas metaheurísticas modernas para optimizar, los algoritmos evolutivos (AE) (\citet{bib-back-handbook}) han surgido como formas flexibles, robustas y eficaces para resolver el problema de planificación de computación heterogénea (o HCSP, del inglés \textit{heterogeneous computing scheduling problem}). Para mejorar aún más la eficiencia de los AEs, las implementaciones paralelas se volvieron una opción popular para agilizar la búsqueda, posibilitando la obtención de resultados de alta calidad en tiempos de ejecución razonables incluso para problemas de optimización difíciles de resolver (\citet{bib-cantu-genetic-algorithms}, \citet{bib-alba-genetic-algorithms}).

\section{Formulación del problema} \label{section:descripcion-problema,subsection:formulacion-problema}

\paragraph{}El contexto del problema se da en un sistema de HC compuesto por varios recursos computacionales o máquinas, para el cual se tiene un conjunto de tareas con requerimientos computacionales variables a ser ejecutadas.
Se considera a una tarea como una unidad atómica de trabajo que puede ser asignada a un recurso computacional; no puede ser dividida en subtareas, y su ejecución no puede ser interrumpida tras ser iniciada.
El hecho de que las máquinas disponibles sean diferentes entre sí en términos de prestaciones, genera una suerte de competencia entre las tareas por ser asignada a la máquina más apta y con mayor rendimiento.
En este trabajo, la única característica que se toma en cuenta sobre una tarea es su tiempo de ejecución por cuestiones de simpleza, aunque claramente se están dejando de lado otras consideraciones que pueden ser cruciales para un sistema real, como puede ser el costo de utilizar un determinado equipo frente a otro; puede suceder que el costo de utilizar un recurso computacional más eficiente que otro sea mayor, dada la demanda que puede llegar a existir sobre dicho recurso, como puede ocurrir en el caso de utilizar recursos computacionales alquilados bajo un formato de infraestructura como servicio (o IaaS del inglés \textit{Infrastructure as a Service}). 
Esto se traduce en que una planificación óptima para la ejecución de tareas será aquella donde el tiempo que tarda la máquina que termina su ejecución por último sea mínimo; este tiempo, como fue mencionado en la sección anterior, es conocido como \textit{makespan}.

\paragraph{}En términos formales, para una instancia del problema de dimensión $X\times Y$, se tiene un conjunto de tareas $T = \{t_1,t_2,\dots,t_X\}$ y un conjunto de recursos computacionales o máquinas $M = \{m_1,m_2,\dots,m_Y \}$. Dada una función de tiempo de ejecución $ET : \{1,\dots,X\} \times \{1,\dots,Y\} \rightarrow R^+$ tal que $ET(i,j)$ es el tiempo requerido para ejecutar la tarea $t_i$ en la máquina $m_j$, el objetivo que se persigue para resolver el problema HCSP es el de encontrar una asignación de tareas a máquinas, dada por una función $f: T^X \rightarrow P^Y$, que minimice el makespan, definido como $max_{m_y \in P} \sum_{t_x \in T: f(t_x)=m_y} ET(t_x, m_y)$.

\paragraph{}Finalmente, se puede expresar que el objetivo de la resolución del problema HCSP se traduce en encontrar o aproximar aquella función que dada un conjunto de máquinas y un conjunto de tareas determine una planificación que minimice el valor del makespan obtenido.

\paragraph{}Cabe destacar que este modelo no toma en cuenta eventuales dependencias entre las tareas, asumiendo que pueden ser ejecutadas de manera independiente. Además, en este trabajo se evalúa la aplicabilidad de un algoritmo de aprendizaje automático particular en el marco de SV en base al estudio de la versión estática del HCSP. Esto quiere decir que las planificaciones tomadas en cuenta se generan de manera previa a la ejecución de un conjunto de tareas, y no se adaptan o varían dinámicamente en tiempo de ejecución.

\section{Complejidad computacional} \label{section:descripcion-problema,subsection:complejidad-computacional}
\section{Estimación de tiempos de ejecución} \label{section:descripcion-problema,subsection:estimacion-tiempo-ejecucion}
\section{Instancias del problema} \label{section:descripcion-problema,subsection:instancias-del-problema}



% se puede construir una matriz con la información del tiempo de ejecución de cada tarea para cada máquina, conocida como matriz \textit{ETC} (del inglés \textit{Expected Time to Compute}). Esta matriz constituye la representación del problema utilizada como entrada para el sistema utilizado en este proyecto de grado.



% \paragraph{}Se utilizó un programa generador de instancias del problema, extraído de \citet{bib-doctorado-nesmachnow}. 

 \section{Heurística de referencia} \label{section:descripcion-problema,subsection:heuristica}
 
\paragraph{}En el algoritmo \ref{algorithm:planificacion-tareas}, como se puede ver en \citet{bib-doctorado-nesmachnow}, se presenta una versión genérica de una heurística generadora de planificaciones frente al problema de HCSP. De manera particular, la heurística utilizada como referencia en este proyecto emplea a Min-Min como criterio para seleccionar tareas en cada iteración del algoritmo. Según este criterio, se escoge de manera \textit{greedy} a la tarea que pueda ser completada primero. Para seleccionar esta tarea, se calcula el mínimo tiempo que cada una de las tareas sin asignar puede demorar en completar para cada una de las máquinas disponibles y se escoge a aquella tarea que en promedio lleve menos tiempo en completar.

\paragraph{}Adicionalmente, para generar instancias del problema en forma de matrices ETC, se utilizó un programa también extraído de \citet{bib-doctorado-nesmachnow}.
%  Se comienza con un conjunto \textit{U} de tareas sin asignar, se calcula el mínimo tiempo que cada una puede demorar en completar para cada máquina, y se escoge a aquella tarea que en promedio lleve menos tiempo en completar. Esta tarea es retirada del conjunto \textit{U}, y el proceso se repite hasta que todas las tareas están asignadas.
 
 % TODO ¿tal vez incluir el código en un apéndice?
 
 \paragraph{}
 \SetEndCharOfAlgoLine{}
 \begin{algorithm}[H]
\SetAlgoLined
\KwIn{conjunto de tareas sin asignar y conjunto de máquinas}
%\KwResult{Task assignment}
 \While{quedan tareas por asignar}{
 	seleccionar tarea de acuerdo a criterio elegido\;
  \For{cada tarea a asignar y cada máquina}{
  	evaluar criterio (tarea, máquina)\;
  }
  asignar tarea seleccionada a máquina seleccionada\;
 }
 \Return{asignación de tareas}
 \caption{Algoritmo genérico de planificación de tareas} \label{algorithm:planificacion-tareas}
\end{algorithm}