\chapter{Introducción} \label{section-introduccion}
\markright{Introducción}

\paragraph{} El Savant Virtual (SV) es un paradigma que mediante la aplicación de técnicas de aprendizaje automático aprende el comportamiento de un algoritmo conocido que resuelve un problema de alto costo computacional. SV genera automáticamente un nuevo programa que resuelve nuevas instancias del problema \cite{savant-original}.
Así también, SV hace un uso más eficiente de los recursos computacionales al posibilitar la paralelización de programas originalmente no paralelos.

\paragraph{} Es de interés encontrar nuevos paradigmas de resolución de problemas NP-difíciles que hagan uso de nuevas técnicas y algoritmos que minimicen los tiempos de búsqueda en los espacios de soluciones asociados.
En este sentido, el enfoque de SV tiene como objetivo obtener soluciones aproximadas a problemas complejos con mayor velocidad de resolución que las heurísticas existentes, manteniendo niveles altos de calidad en las soluciones. 

\paragraph{} En este proyecto se pretende evaluar el desempeño y la aplicabilidad de las redes neuronales como clasificadores de aprendizaje automático, presentando una alternativa a la utilización de máquinas de soporte vectorial (SVM), contribuyendo a la investigación que dio origen a este paradigma.
Con este objetivo, se estudió el rendimiento de las redes neuronales en el marco del estudio del problema \textit{Heterogeneous Computing Scheduling Problem} (HCSP), un problema de optimización combinatoria NP-difícil.
Se llevó adelante un estudio comparativo entre redes neuronales y SVM entrenando diferentes configuraciones de redes neuronales, y se compararon las soluciones obtenidas con las soluciones generadas por SVM. 

\newpage % orphaned line.

\paragraph{} Se encontraron mejoras en las medidas de desempeño seleccionadas para grandes dimensiones del problema, como se puede ver en el Capítulo \ref{capitulo:analisis-experimental}.

\paragraph{}Las principales contribuciones de este proyecto de grado son:
\begin{enumerate}
    \item Extender el trabajo original de SV presentado por \citet{savant-original}.
    \item Analizar el comportamiento de SV cuando se utilizan redes neuronales como método de aprendizaje computacional.
    \item Cuantificar el rendimiento de las redes neuronales en comparación con SVM, utilizando el \textit{makespan} del problema HCSP como medida principal de la calidad de las soluciones generadas.
    \item Explorar las configuraciones de redes neuronales más adecuadas para resolver el problema HCSP.
\end{enumerate}

\paragraph{} El resto del trabajo se estructura como se explica a continuación.
En el Capítulo \ref{section-descripcion-problema} se hace una presentación del problema HCSP y la heurística de referencia. El Capítulo \ref{section-marco-teorico} presenta el marco teórico en el cual se basó este trabajo, enfocado en aprendizaje automático y en los dos algoritmos comparados durante el trabajo.
El Capítulo \ref{section-trabajo-relacionado} aborda los trabajos relacionados, donde se presentan aquellos trabajos asociados a SV y a técnicas de clasificación y selección de atributos que sirvieron de referencia o punto de partida para este trabajo.
El Capítulo \ref{chapter-implementation} presenta la implementación del sistema utilizado para llevar a cabo la experimentación asociada a este trabajo, dejando para el Capítulo \ref{capitulo:analisis-experimental} el análisis experimental.
Por último, el Capítulo \ref{section-conclusiones} presenta las conclusiones y trabajo futuro.