\chapter{Introducción} \label{section-introduccion}
\markright{Introducción}

\paragraph{} El Savant Virtual o SV es un paradigma que mediante la aplicación de técnicas de aprendizaje automático aprende el comportamiento de un algoritmo conocido que resuelve un problema y genera automáticamente un nuevo programa que resuelve nuevas instancias del problema de manera aproximada al programa original, incluso para problemas de tamaño diferente.
SV aprende de las heurísticas que tradicionalmente resuelven estos problemas de interés, pudiendo hacer un uso más eficiente de los recursos al posibilitarse la paralelización de programas originalmente no paralelos.

\paragraph{} Es de interés encontrar nuevos paradigmas de resolución de problemas NP-difíciles que hagan uso de nuevas técnicas y algoritmos que minimicen los tiempos de búsqueda en espacios de soluciones.
En este sentido, el enfoque de SV tiene como objetivo obtener soluciones a problemas complejos con mayor velocidad de resolución que las heurísticas existentes, manteniendo niveles altos de calidad en las soluciones. 

\paragraph{} En este proyecto se pretende evaluar el desempeño y aplicabilidad de las redes neuronales como clasificador de aprendizaje automático, presentando una alternativa a la utilización de máquinas de soporte vectorial (SVM), contribuyendo a la investigación que dio origen a este paradigma.
Para esto, se estudia su rendimiento en el marco del estudio del problema \textit{Heterogeneous Computing Scheduling Problem}, un problema de optimización combinatoria NP-difícil.
Se llevó adelante un estudio comparativo entre redes neuronales y SVM, entrenando diferentes configuraciones de redes neuronales y se compararon las soluciones obtenidas con las soluciones generadas por SVM, encontrando mejoras en las medidas de desempeño seleccionadas para grandes dimensiones del problema.

\paragraph{} El resto del trabajo se estructura como se explica a continuación.
En el Capítulo \ref{section-descripcion-problema} se hace una presentación del problema \textit{Heterogeneous Computing Scheduling Problem} y la heurística de referencia, el Capítulo \ref{section-marco-teorico} presenta marco teórico dentro del cual se trabajará, enfocado en aprendizaje automático y los dos algoritmos que se comparan durante el trabajo.
El Capítulo \ref{section-trabajo-relacionado} aborda los trabajos relacionados, donde se presentan aquellos trabajos relacionados a SV así como técnicas de clasificación y selección de atributos.
El Capítulo \ref{chapter-implementation} presenta la implementación de software que se utiliza para construir los experimentos, dejando para el Capítulo \ref{capitulo:analisis-experimental} el análisis experimental.
Por último, el Capítulo \ref{section-conclusiones} presenta las conclusiones y trabajo a futuro.
