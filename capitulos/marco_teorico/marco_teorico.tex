\chapter{Marco teórico} \label{section-marco-teorico}
\markright{Marco teórico}

\paragraph{} A continuación, se presentarán los conceptos fundamentales que apoyan al desarrollo y análisis experimental llevado a cabo en este proyecto de grado, involucrando principalmente conceptos de aprendizaje automático.

\section{Aprendizaje automático}

\paragraph{}Aprendizaje automático es un campo de las ciencias de la computación y una rama de la inteligencia artificial dedicada a desarrollar técnicas que permitan construir programas que mejoran de forma automática basados en la experiencia, se dice que estos programas \textit{aprenden} con la experiencia. En términos formales, el concepto de que un programa aprenda es puesto en palabras por \citet{mitchell1997machine} de la siguiente mantera: \textit{“Se dice que un programa de computadora aprende de la experiencia E con respecto a una tarea T y medida de desempeño P si su desempeño en la tarea T, medida en términos de P, mejora con la experiencia E”}. 

\paragraph{}Los algoritmos de aprendizaje automático pueden ser clasificados en algoritmos de aprendizaje supervisado y no supervisado en función de la necesidad de la existencia de datos previos que oficien de experiencias. En aprendizaje supervisado las entradas del algoritmo son ejemplos de experiencias (conocidos como datos de entrenamiento) y sus respectivas salidas esperadas, de tal manera, el algoritmo aprende reglas generales de mapeo entre entradas y salidas. Ejemplos de algoritmos de aprendizaje supervisado son las \textit{redes neuronales con propagación hacia atrás} , así como las \textit{máquinas de soporte vectorial} o los \textit{árboles de decisión}, entre otros. En algoritmos no supervisados no se ofrecen las salidas esperadas en los datos de entrenamiento, dejando a los algoritmos detectar patrones en los datos por sí mismos; el algoritmo \textit{K-Means} es un ejemplo de aprendizaje automático no supervisado. 

\paragraph{}También podemos encontrar algoritmos de aprendizaje por refuerzo. Dentro del aprendizaje automático, el aprendizaje por refuerzo se ocupa de estudiar cómo agentes de software toman acciones en su entorno en con el objetivo de maximizar una recompensa. Un ejemplo de este tipo de algoritmos es \textit{Q-Learning}, que involucra el aprendizaje de una política que le indique al agente qué decisión tomar bajo las circunstancias en las que se encuentre.

\paragraph{}Los problemas estudiados en aprendizaje automático pueden ser de clasificación o de regresión. Un problema se considera de clasificación cuando se puede clasificar a las instancias del problema de acuerdo a valores de un dominio discreto, como en el ejemplo de reconocimiento de flores mediante imágenes, donde la clasificación resulta binaria (una imagen representa a una flor o no lo hace). Un problema se considera de regresión cuando se puede clasificar a las instancias del problema de acuerdo a valores de un dominio continuo, por ejemplo al intentar predecir el valor de un inmueble dado un conjunto de características asociadas al mismo.

\paragraph{}Las técnicas basadas en aprendizaje automático han tomando mayor relevancia en los últimos años debido al crecimiento de los volúmenes y variedades de datos, disminución de los costos de infraestructura, acompañado por un crecimiento en poder computacional y capacidad de almacenamiento a precios accesibles.

\paragraph{}En este proyecto trabajaremos con dos algoritmos de aprendizaje automático conocidos como \textit{máquina de soporte vectorial} y \textit{redes neuronales}, en un problema de clasificación.

\section{Clasificadores de interés}

\subsection{Redes neuronales artificiales}

\paragraph{}Una red neuronal artificial es un algoritmo o clasificador de aprendizaje automático, utilizada tanto para problemas de clasificación como de regresión.
Su concepción está inspirada en parte por la observación de los mecanismos de aprendizaje en sistemas biológicos. Las redes neuronales artificiales están formadas por unidades más simples e interconectadas, donde cada unidad tiene entradas reales y produce una única salida también real, mediante asignación de pesos en regresiones lineales.
Los perceptrones son las unidades sobre las cuales están construidos los sistemas de redes neuronales, dado que por sí solos únicamente puede expresar decisiones lineales, pero su composición en redes neuronales multicapa puede expresar una variedad de funciones objetivo no lineales. 
Un perceptrón recibe valores de $n$ entradas $x_1, x_2, \dots x_n$, realiza una combinación lineal con ellas obteniendo una expresión $x_1 w_1 + x_2 w_2 + \dots x_n w_n$ donde $w_i$ es el peso otorgado a $x_i$, para todo $i \in \{1, \dots, n\}$. Esta expresión es evaluada con una función conocida como \textit{función de activación}, que devuelve un valor de acuerdo a si se supera un umbral determinado o no con la combinación lineal de los valores de entrada. Por ejemplo, si se utiliza a la función signo como función de activación, si la combinación lineal de los valores de entrada supera a un umbral de activación definido por el perceptrón, se devolverá $1$ como salida y $-1$ en caso contrario. La figura \ref{fig:perceptron} muestra la estructura de dicho perceptrón.

\textsc{\begin{figure}[ht!]
	\centering
    \includegraphics[width=.5\linewidth]{imagenes/perceptron.png}
	\caption{Perceptron con función de activación \textit{f = signo(x)}}
	\label{fig:perceptron}
\end{figure}}

\paragraph{}Cuando se calcula la combinación lineal de los valores de entrada, su valor resultante puede oscilar entre $-\infty$ y $+\infty$, dado que desde un perceptrón no se posee una referencia de cuáles son los límites presentes en las posibles clases de clasificación para el problema de interés. Con el propósito de limitar este valor para producir la salida de un perceptrón es que se usan las funciones de activación. Las funciones de activación utilizadas en este proyecto se detallan a continuación.
La función tanh, expresada de la siguiente manera: $f(x) = tanh(x) = \frac{2}{1 + e^{-2x}} - 1 $ es una función continua no lineal y al componerla consigo misma se obtienen funciones no lineales, lo que permite combinar a unidades o perceptrones con esta función de activación sin perder esa no linealidad. Es una función suave en su curva, mostrando que pequeñas variaciones en valores del dominio cercanos a 0 generan cambios grandes en los valores correspondientes del codominio; esto implica que se le da una gran ponderación a los valores de los extremos del codominio, algo análogo a una tasa de aprendizaje.
La función relu, expresada de la siguiente manera: $f(x) = relu(x) = max(0, x)$ es no lineal y las composiciones de ella consigo misma serán no lineales, pero por su forma, puede generar que algunas neuronas den como resultado cero, constituyéndose una eventual pérdida de información.
Este problema se llama \textit{dying relu problem}. Así también, computar relu es menos costoso que computar tanh porque implica operaciones matemáticas más simples.
La función identity, también llamada de activación lineal, expresada de la siguiente manera: $f(x) = identity(x) = x$, siempre retorna el mismo valor que recibe en su argumento, lo que implica que equivale a una regresión lineal utilizando los pesos de la unidad.

% Backpropagation
\paragraph{}En relación al entrenamiento de las redes neuronales, en este proyecto de grado se utiliza el algoritmo Backpropagation para ajustar los pesos que cada perceptrón de la red asocia a cada una de sus entradas. Dado un nuevo ejemplo de entrenamiento, se generan las salidas que correspondan de la red neuronal, se comparan con las salidas esperadas y si el error es mayor al deseado, se realiza una pasada "hacia atrás" por todos los perceptrones para ajustar sus pesos y se intenta clasificar al ejemplo de entrenamiento nuevamente. Esto se da repetidamente para cada ejemplo de entrenamiento hasta obtener un error aceptable en comparación con la salida esperada de la red neuronal.

% Ventajas y desventajas
\paragraph{}Entre las ventajas de utilizar redes neuronales se encuentran el hecho de que se adecuan correctamente a problemas en los cuales los datos de entrenamiento contienen ruido y en contextos en los cuales tiempos largos de entrenamiento son aceptables. Además, suelen mantener tiempos bajos en clasificación.
Además, como se menciona en la sección \ref{section-trabajos-tecnicas-clasificacion}, una desventaja de las redes neuronales es que su representación interna es difícil de entender para los humanos y por este motivo se dice que se comportan como una “caja negra”.

\subsection{SVM}

\paragraph{}Una SVM es un algoritmo o clasificador de aprendizaje automático supervisado utilizado fundamentalmente para problemas de clasificación.
Dado un problema de aprendizaje automático supervisado de clasificación donde las instancias del problema pueden ser clasificadas en $N$ clases y un conjunto de ejemplos de entrenamiento de dicho problema, una SVM busca encontrar un hiperplano que los divida de manera lineal en esas $N$ clases. De esta manera, frente a una nueva instancia del problema, la SVM será capaz de clasificarla generando una correspondencia entre esta instancia y una de las $N$ clases posibles.

\paragraph{}Se denominan vectores de soporte a aquellos ejemplos de entrenamiento más cercanos al hiperplano y esta denominación se refiere a que sin su presencia el hiperplano podría no ser el mismo, ya que podría utilizarse uno más óptimo. Esta distancia entre el hiperplano y un vector de soporte se conoce como margen y para un conjunto de ejemplos de entrenamiento, SVM busca dividirlos de manera óptima con un hiperplano donde se maximicen estos márgenes para cada uno de los vectores de soporte, cada uno asociado a su vez a una de las posibles clases de clasificación del problema.
En un problema donde las instancias pueden ser clasificadas en dos clases, el hiperplano de dos dimensiones es representado por una línea recta y los vectores de soporte son aquellos ejemplos de entrenamiento más cercanos a ella, desde las dos direcciones posibles. La confianza en la clasificación de una instancia del problema, por tanto, estará dada por la distancia de dicho instancia al hiperplano de manera directamente proporcional.
La obtención de un hiperplano que divida de manera lineal a los ejemplos de entrenamiento no siempre es posible y se suele utilizar una técnica conocida como \textit{kernelización} cuando esto sucede, para aumentar la dimensión del dominio donde se está buscando un hiperplano e intentar obtener un hiperplano que separe linealmente a los ejemplos de entrenamiento en esa dimensión. De esta manera, es posible clasificar a nuevas instancias del problema de acuerdo a este nuevo hiperplano y transformar la clasificación obtenida a la dimensión original del problema dada por las clases de clasificación disponibles.

\paragraph{}Entre las ventajas de utilizar SVM se encuentran que funciona bien en conjuntos de datos pequeños y limpios (con datos linealmente separables), resultando eficiente ya que utiliza un subconjunto de los ejemplos de entrenamiento únicamente, asociados a la definición del hiperplano. Las desventajas incluyen la poca efectividad que tienen en conjuntos de datos con ruido o clases diferenciadas de manera poco clara y que los tiempos de entrenamiento asociados pueden ser significativos para grandes conjuntos de entrenamiento.

