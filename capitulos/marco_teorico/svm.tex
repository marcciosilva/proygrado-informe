\subsection{SVM}

\paragraph{}Una SVM es un algoritmo o clasificador de aprendizaje automático supervisado utilizado fundamentalmente para problemas de clasificación.
Dado un problema de aprendizaje automático supervisado de clasificación donde las instancias del problema pueden ser clasificadas en $N$ clases y un conjunto de ejemplos de entrenamiento de dicho problema, una SVM busca encontrar un hiperplano que los divida de manera lineal en esas $N$ clases. De esta manera, frente a una nueva instancia del problema, la SVM será capaz de clasificarla generando una correspondencia entre esta instancia y una de las $N$ clases posibles.

\paragraph{}Se denominan vectores de soporte a aquellos ejemplos de entrenamiento más cercanos al hiperplano y esta denominación se refiere a que sin su presencia el hiperplano podría no ser el mismo, ya que podría utilizarse uno más óptimo. Esta distancia entre el hiperplano y un vector de soporte se conoce como margen y para un conjunto de ejemplos de entrenamiento, SVM busca dividirlos de manera óptima con un hiperplano donde se maximicen estos márgenes para cada uno de los vectores de soporte, cada uno asociado a su vez a una de las posibles clases de clasificación del problema.
En un problema donde las instancias pueden ser clasificadas en dos clases, el hiperplano de dos dimensiones es representado por una línea recta y los vectores de soporte son aquellos ejemplos de entrenamiento más cercanos a ella, desde las dos direcciones posibles. La confianza en la clasificación de una instancia del problema, por tanto, estará dada por la distancia de dicho instancia al hiperplano de manera directamente proporcional.
La obtención de un hiperplano que divida de manera lineal a los ejemplos de entrenamiento no siempre es posible y se suele utilizar una técnica conocida como \textit{kernelización} cuando esto sucede, para aumentar la dimensión del dominio donde se está buscando un hiperplano e intentar obtener un hiperplano que separe linealmente a los ejemplos de entrenamiento en esa dimensión. De esta manera, es posible clasificar a nuevas instancias del problema de acuerdo a este nuevo hiperplano y transformar la clasificación obtenida a la dimensión original del problema dada por las clases de clasificación disponibles.

\paragraph{}Entre las ventajas de utilizar SVM se encuentran que funciona bien en conjuntos de datos pequeños y limpios (con datos linealmente separables), resultando eficiente ya que utiliza un subconjunto de los ejemplos de entrenamiento únicamente, asociados a la definición del hiperplano. Las desventajas incluyen la poca efectividad que tienen en conjuntos de datos con ruido o clases diferenciadas de manera poco clara y que los tiempos de entrenamiento asociados pueden ser significativos para grandes conjuntos de entrenamiento.