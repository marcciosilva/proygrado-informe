\begin{abstract}
    \paragraph{} Este trabajo estudia el rendimiento del método de aprendizaje automático conocido como redes neuronales en el marco del paradigma Savant Virtual o SV. Bajo este paradigma, un sistema de software aprende a resolver problemas de alto costo computacional utilizando técnicas de aprendizaje automático.
    \paragraph{} SV toma como referencia el comportamiento de las heurísticas que tradicionalmente se utilizan para resolver estos problemas.
    \paragraph{} En particular se extiende el trabajo realizado por \citet{savant-original}, haciendo un estudio comparativo entre máquinas de soporte vectorial y redes neuronales como métodos de aprendizaje automático asociados al paradigma.
    \paragraph{} Para llevar a cabo este estudio, se implementaron 3 clasificadores basados en redes neuronales con distintas arquitecturas y funciones de activación, y un clasificador de máquina de soporte vectorial.
    \paragraph{} Este trabajo se enfoca en la resolución del problema conocido como \textit{Heterogeneous Computing Scheduling Problem}, utilizando, entre otras, la precisión de los clasificadores como indicador del éxito respectivo de cada uno.
    \paragraph{} Los resultados experimentales mostraron que para determinadas configuraciones de las redes neuronales, se mejoran los resultados de la máquina de soporte vectorial de acuerdo a las métricas de éxito consideradas. 
\end{abstract}