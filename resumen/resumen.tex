\begin{abstract}
    \paragraph{}Este trabajo estudia el comportamiento del paradigma Savant Virtual el cual, mediante la aplicación de métodos de aprendizaje computacional, permite resolver problemas de optimización. Savant Virtual aprende de los algoritmos que tradicionalmente se utilizan para resolver el problema que se desea abordar. Este proyecto de grado presenta un estudio comparativo entre máquinas de soporte vectorial (SVM) y redes neuronales como métodos de aprendizaje automático asociados al paradigma Savant Virtual. 
    Con este propósito se implementan tres clasificadores basados en redes neuronales, variando las funciones de activación, y un clasificador SVM. El problema de optimización abordado es el \textit{Heterogeneous Computing Scheduling Problem}. Los clasificadores se entrenan con 100 instancias del problema de 512 tareas y 16 máquinas, lo que se traduce en 512000 instancias de entrenamiento. La evaluación experimental se realiza sobre instancias del problema en un rango de dimensiones que va desde 17 tareas y 16 máquinas hasta 1024 tareas y 16 máquinas, con el fin de analizar la escalabilidad del Savant Virtual. Se utiliza el \textit{makespan} para evaluar las soluciones halladas con los distintos clasificadores y también se analiza la precisión en la clasificación. Los resultados experimentales muestran que, para determinadas configuraciones de las redes neuronales, el \textit{makespan} mejora con respecto a las soluciones obtenidas con SVM. De igual forma, se constatan mejoras en las redes neuronales sobre SVM al comparar los resultados alcanzados en términos de la precisión de las predicciones.
\end{abstract}