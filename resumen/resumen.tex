\begin{abstract}
    \paragraph{}Este trabajo estudia el comportamiento del paradigma Savant Virutal el cual, mediante la aplicación de métodos de aprendizaje computacional, aprende a resolver problemas de alto costo computacional. Savant Virtual aprende de las heurísticas que tradicionalmente se utilizan para resolver estos problemas. En particular se extiende el trabajo realizado por \citet{savant-original}, haciendo un estudio comparativo entre Máquinas de Soporte Vectoria (SVM) y Redes Neuronales como métodos de aprendizaje automático asociados al paradigma. 
    Para esto se implementan 3 clasificadores basados en redes neuronales variando las funciones de activación, y un clasificador de máquina de soporte vectorial. Los clasificadores se entrenan con 100 instancias del problema de 512 tareas y 16 máquinas, lo que se traduce en 512000 instancias de entrenamiento. Se clasifican instancias del problema en un rango de dimensiones que va desde 17 tareas y 16 máquinas hasta 1024 tareas y 16 máquinas, con el fin de analizar la escalabilidad del SV. El problema sobre el cual se trabaja es el problema de \textit{Heterogeneous Computing Scheduling Problem}. Se utiliza el \textit{makespan} como indicador principal de performance del sistema, también se analiza la precisión en clasificación. Los resultados experimentales muestran que para determinadas configuraciones de las redes neuronales, el \textit{makespan} mejora con respecto a SVM. También se encuentran mejoras en la precisión de las redes neuronales con respecto a SVM.
\end{abstract}